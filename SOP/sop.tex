\documentclass[a4paper,11pt]{article}

% Padding Requirements
\usepackage[a4paper, margin=1in]{geometry}
% Used for images
\usepackage{graphicx}
\usepackage{float}
\graphicspath{{/Users/sloth_mini/Documents/4.1/dissertation/images/}}
% Used for bibliography
\usepackage[style=ieee, sorting=none, backend=biber]{biblatex}
\renewcommand{\bibfont}{\small}

% References file
\addbibresource{references.bib}


% ---- UNIVERSITY NAME ---- %
\newcommand{\uni}{McMaster University}
\newcommand{\city}{Hamilton}
\newcommand{\program}{MSc. Computer Science }
% ----- SUPERVISOR ----- %
\newcommand{\supervisor}{\textbf{Prof. Neamat El Gayar}}

% ---- DISSERTATION TITLE ---- %
\newcommand{\project}{Multi-Element Association Rule Derivation from Minimum Spanning Trees}

% ----- HERIOT WATT UNVIERSITY (DUBAI?) ---- %
\newcommand{\hw}{Heriot-Watt University}

\begin{document}
\title{Statement of Purpose}

\author{Sahil Pattni\\(sp72@hw.ac.uk)\\\\Applying to \program  at \uni}
\date{}
\maketitle

% Introduction and purpose of letter.
Few topics have fostered such a sense of curiosity, wonder and interest as [SUBJECT]. It is for this reason that I want to pursue a Master in Computer Science at \uni. My research interests include machine learning, data structures and natural language processing.

% Motivation for pursuing research.
My motivation for pursuing research relating to machine learning stems from building my own neural network from scratch and optimizing it using Particle Swarm Optimization for my Biologically Inspired Computation course. It was after building this model that I truly felt the power that machine learning can offer, and it was then that I decided I wanted to learn as much as I could about the work that had been done in this domain, and to contribute towards its advance.

% Dissertation.
I am working on a my undergraduate dissertation under the guidance of \supervisor, the purpose of which is to derive multi-element association rules by representing, as a graph, the association between product nodes as an edge that connects them, extracting the minimum spanning tree such that only the strongest associations remain, and determining clusters from this resulting minimum spanning tree using Markov Clustering. From each cluster, multi-element rules can be obtained from the edges within the cluster. This is an alternative to the established Apriori algorithm \cite{apriori}. I am building on the work of \textbf{Marucio A. Valle et al} \cite{mst_paper}, who proposed the initial framework for single element association rule derivation from minimum spanning trees. This dissertation has allowed me to explore my interest in data structures and graph theory, but it has come with its share of technical challenges, all of which I enjoyed solving. Transforming the large dataset I am working on was computationally infeasable if performed linearly with a complexity of $O(n^2)$, due to the millions of entries in the dataset. I was able to use the knowledge I had gained in courses such as Operating Systems and Concurrency to develop a multi-core, parallelized solution to meet these challenges.

% Undergraduate career.
In the four years it took me to achieve a Bachelor in Science with Honors at \hw, I have acquired an understanding of this scientific discipline that has only fueled my curiosity and interest further. I have worked on several interesting projects - from using C and ARM Assembly to interact with hardware connected to a Raspberry Pi to writing our own parsers for compilers, and classifying the sentiments of tweets on a five point scale using LSTM RNNs. It was the latter that sparked my interest in natural language processing (NLP). Diving deeper into the domain, I found generative language models such as GPT-2 to be utterly fascinating, coming one step closer towards bridging the gap that divides man and machine.
% Mitigating circumstances.
Towards the end of the second semester of my third academic year, the world was hit with the Covid pandemic, and as universities, schools and businesses were closed under a government-mandated lockdown, my university did not have enough time to transition to an online examination system. Therefore, for courses whose grades were primarily determined through exams, a $P$ (passing) grade was awarded since the exams could not be administered. For courses that were marked through courseworks, the regular grading system was used as all courseworks had been submitted by the time the lockdown was put in effect.

Outside the academic environment, my passion for computer science has driven me to work on several projects on the side; most recently, I worked with two colleagues to develop a cross-platform mobile application for a hackathon that allows non-profits to crowd-source their funding. I am also currently developing an automated cryptocurrency trading system that uses machine learning\footnote{I am still comparing different classifiers and architectures to determine which one would best be suited for this task} to determine when to execute trades. 

At \uni, due to the plethora of available opportunities, I believe that I will be able to gain higher exposure and with it, a deeper understanding of the various problems in the field. I would specifically like to focus on machine learning and NLP.

\printbibliography
\end{document}